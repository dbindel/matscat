\documentclass{article}
\usepackage{amsmath,amssymb}

\newcommand{\schrodinger}{Schr\"{o}dinger}
\newcommand{\calL}{\mathcal{L}}
\newcommand{\bbR}{\mathbb{R}}
\newcommand{\bbC}{\mathbb{C}}

\begin{document}

These notes describe some aspects of potential scattering on the real line
and how scattering coefficients, scattered wave functions, and resonances.

\section{Background}

We consider the one-dimensional time-independent \schrodinger\ equation
\begin{equation}
  H \psi = E \psi
  \label{basic-schrodinger}
\end{equation}
where
\begin{equation}
  H = -\frac{d^2}{dx^2} + V(x).
\end{equation}
For these notes, we will assume the potential $V$ is piecewise
continuous and has compact support contained within some interval
$[a,b]$.  For discrete values of $E$ in $(-\infty,0)$, there are
nontrivial solutions to (\ref{basic-schrodinger}) in $L^2(\bbR)$;
these are the \emph{bound states}.  In addition to the bound states,
(\ref{basic-schrodinger}) admits solutions in $L^{\infty}$ for $E \in
[0,\infty)$; the properties of these \emph{unbound states} are our
main concern.

For any $E \in \bbC$, (\ref{basic-schrodinger}) has a two-dimensional
space of (weakly) twice-differentiable solutions.  If we write $k^2 =
-E$, then we have
\begin{equation}
  \psi(x) = \left\{
    \begin{array}{ll}
    A_l \exp(ikx) + A_r \exp(-ikx), & \mbox{ for } x < a \\
    B_l \exp(ikx) + B_r \exp(-ikx), & \mbox{ for } x > b \\
    \end{array}
  \right.
\end{equation}
When $k$ is on the positive real axis, $A_l$ and $B_l$ are the magnitudes
of the left-traveling part of the wave in the regions outside the potential,
while $A_r$ and $B_r$ are the magnitudes of the right-traveling waves.
Alternately, we can view $A_r$ and $B_l$ as the incoming wave magnitudes,
and $A_l$ and $B_r$ the outgoing wave magnitudes.  The incoming wave
components are related to the outgoing wave components by the
scattering matrix $S(k)$:
\begin{equation}
  \begin{bmatrix} A_l \\ B_r \end{bmatrix} =
  S(k)
  \begin{bmatrix} A_r \\ B_l \end{bmatrix}.
\end{equation}

For $k$ on the positive real line, $S(k)$ can be interpreted as a
relation between incoming waves and outgoing waves.  It is a unitary
matrix whose entries can be assigned physical meanings: $S_{11}(k)$
and $S_{12}(k)$ are the amplitude of the reflected and transmitted
waves, respectively, in response to an incoming wave from $-\infty$;
while $S_{22}(k)$ and $S_{21}(k)$ give the amplitude of the reflected
and transmitted waves in response to a wave from $+\infty$.  For most
$k \in \bbC$, $S(k)$ is well-defined and analytic with respect to $k$,
though for non-real $k$ the matrix will not be unitary and the
physical interpretation of the coefficients is questionable.  However,
the singularities of $S(k)$ in the complex plane provide useful
information about the behavior of $S(k)$.  In the upper half plane
$\Im(k) \geq 0$, $S(k)$ may only have poles along the imaginary axis;
these correspond to bound state energies for the \schrodinger\
equation (\ref{basic-schrodinger}).  The poles of $S(k)$ in the lower
half plane are known as resonance poles.

% I'm confuseed now why Maciej was confused that there should be
% spikes in the transmission coefficient near resonances.  There
% should certainly be *some* rapid variation in behavior at those
% points; I'm not sure I see why the variation should always be a
% spike in the reflection coefficient or in the transmission
% coefficient.

For $\lambda^2 = E > 0$, solutions to (\ref{basic-schrodinger}) satisfy
\begin{eqnarray*}
  \psi(x) & = & A_{-} e^{i \lambda x} + B_{-} e^{-i \lambda x}, \;
    x \in (-\infty,a] \\
  \psi(x) & = & A_{+} e^{i \lambda x} + B_{+} e^{-i \lambda x}, \;
    x \in [b,\infty)
\end{eqnarray*}
The solution components proportional to $e^{-i \lambda x}$ correspond
to right-traveling waves in the time-dependent equations, while the
solution components proportional to $e^{i \lambda x}$ correspond to
left-traveling waves.  In general, we can freely choose any two of the
coefficients $A_{-}$, $A_{+}$, $B_{-}$, and $B_{+}$; the remaining two
coefficients are then determined by the constraint that $\psi$ satisfies
the \schrodinger\ equation.  For scattering calculations, we typically
choose the incoming wave coefficients $B_{-}$ and $A_{+}$, and from them
compute the outgoing wave coefficients $A_{-}$ and $B_{+}$.  The
\emph{scattering matrix} $S(\lambda)$ relates the two sets of coefficients:
\begin{equation}
  \begin{bmatrix} A_{-} \\ B_{+} \end{bmatrix} = 
  S(\lambda)
  \begin{bmatrix} B_{-} \\ A_{+} \end{bmatrix}.
\end{equation}

When a right-traveling plane wave interacts with the potential on $[a,b]$,
some of that plane wave will be reflected and some will be transmitted.
To describe this interaction, we compute the reflected wave coefficient
$A_{-}$ and the transmitted wave coefficient $B_{+}$ given an incident
wave of unit magnitude from $-\infty$ (i.e. $B_{-} = 1$) and no wave
from $+\infty$ (i.e. $A_{+} = 0$).  The reflected 


% The differential equation
% Scattering of a plane wave
% Homogeneous solutions
% Bound states and resonance poles

% We consider the one-dimensional time-harmonic \schrodinger\ equation 
% \begin{equation}
%   \left( -frac{d^2}{dx^2} + V(x) - \lambda^2 \right) \psi = 0,
% \end{equation}
% where $V$ is supported in a bounded interval $[a,b]$.  

% \begin{equation}
%   H_0 = -\frac{d}{dx^2}
% \end{equation}

% \begin{equation}
%   H = H_0 + V(x)
% \end{equation}

% \begin{equation}
%   (H_0 - \lambda^2) \psi_0 = 0
% \end{equation}

% \begin{equation}
  
% \end{equation}

\end{document}